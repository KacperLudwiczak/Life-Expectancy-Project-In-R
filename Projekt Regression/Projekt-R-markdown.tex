% Options for packages loaded elsewhere
\PassOptionsToPackage{unicode}{hyperref}
\PassOptionsToPackage{hyphens}{url}
%
\documentclass[
]{article}
\usepackage{amsmath,amssymb}
\usepackage{lmodern}
\usepackage{iftex}
\ifPDFTeX
  \usepackage[T1]{fontenc}
  \usepackage[utf8]{inputenc}
  \usepackage{textcomp} % provide euro and other symbols
\else % if luatex or xetex
  \usepackage{unicode-math}
  \defaultfontfeatures{Scale=MatchLowercase}
  \defaultfontfeatures[\rmfamily]{Ligatures=TeX,Scale=1}
\fi
% Use upquote if available, for straight quotes in verbatim environments
\IfFileExists{upquote.sty}{\usepackage{upquote}}{}
\IfFileExists{microtype.sty}{% use microtype if available
  \usepackage[]{microtype}
  \UseMicrotypeSet[protrusion]{basicmath} % disable protrusion for tt fonts
}{}
\makeatletter
\@ifundefined{KOMAClassName}{% if non-KOMA class
  \IfFileExists{parskip.sty}{%
    \usepackage{parskip}
  }{% else
    \setlength{\parindent}{0pt}
    \setlength{\parskip}{6pt plus 2pt minus 1pt}}
}{% if KOMA class
  \KOMAoptions{parskip=half}}
\makeatother
\usepackage{xcolor}
\usepackage[margin=1in]{geometry}
\usepackage{color}
\usepackage{fancyvrb}
\newcommand{\VerbBar}{|}
\newcommand{\VERB}{\Verb[commandchars=\\\{\}]}
\DefineVerbatimEnvironment{Highlighting}{Verbatim}{commandchars=\\\{\}}
% Add ',fontsize=\small' for more characters per line
\usepackage{framed}
\definecolor{shadecolor}{RGB}{248,248,248}
\newenvironment{Shaded}{\begin{snugshade}}{\end{snugshade}}
\newcommand{\AlertTok}[1]{\textcolor[rgb]{0.94,0.16,0.16}{#1}}
\newcommand{\AnnotationTok}[1]{\textcolor[rgb]{0.56,0.35,0.01}{\textbf{\textit{#1}}}}
\newcommand{\AttributeTok}[1]{\textcolor[rgb]{0.77,0.63,0.00}{#1}}
\newcommand{\BaseNTok}[1]{\textcolor[rgb]{0.00,0.00,0.81}{#1}}
\newcommand{\BuiltInTok}[1]{#1}
\newcommand{\CharTok}[1]{\textcolor[rgb]{0.31,0.60,0.02}{#1}}
\newcommand{\CommentTok}[1]{\textcolor[rgb]{0.56,0.35,0.01}{\textit{#1}}}
\newcommand{\CommentVarTok}[1]{\textcolor[rgb]{0.56,0.35,0.01}{\textbf{\textit{#1}}}}
\newcommand{\ConstantTok}[1]{\textcolor[rgb]{0.00,0.00,0.00}{#1}}
\newcommand{\ControlFlowTok}[1]{\textcolor[rgb]{0.13,0.29,0.53}{\textbf{#1}}}
\newcommand{\DataTypeTok}[1]{\textcolor[rgb]{0.13,0.29,0.53}{#1}}
\newcommand{\DecValTok}[1]{\textcolor[rgb]{0.00,0.00,0.81}{#1}}
\newcommand{\DocumentationTok}[1]{\textcolor[rgb]{0.56,0.35,0.01}{\textbf{\textit{#1}}}}
\newcommand{\ErrorTok}[1]{\textcolor[rgb]{0.64,0.00,0.00}{\textbf{#1}}}
\newcommand{\ExtensionTok}[1]{#1}
\newcommand{\FloatTok}[1]{\textcolor[rgb]{0.00,0.00,0.81}{#1}}
\newcommand{\FunctionTok}[1]{\textcolor[rgb]{0.00,0.00,0.00}{#1}}
\newcommand{\ImportTok}[1]{#1}
\newcommand{\InformationTok}[1]{\textcolor[rgb]{0.56,0.35,0.01}{\textbf{\textit{#1}}}}
\newcommand{\KeywordTok}[1]{\textcolor[rgb]{0.13,0.29,0.53}{\textbf{#1}}}
\newcommand{\NormalTok}[1]{#1}
\newcommand{\OperatorTok}[1]{\textcolor[rgb]{0.81,0.36,0.00}{\textbf{#1}}}
\newcommand{\OtherTok}[1]{\textcolor[rgb]{0.56,0.35,0.01}{#1}}
\newcommand{\PreprocessorTok}[1]{\textcolor[rgb]{0.56,0.35,0.01}{\textit{#1}}}
\newcommand{\RegionMarkerTok}[1]{#1}
\newcommand{\SpecialCharTok}[1]{\textcolor[rgb]{0.00,0.00,0.00}{#1}}
\newcommand{\SpecialStringTok}[1]{\textcolor[rgb]{0.31,0.60,0.02}{#1}}
\newcommand{\StringTok}[1]{\textcolor[rgb]{0.31,0.60,0.02}{#1}}
\newcommand{\VariableTok}[1]{\textcolor[rgb]{0.00,0.00,0.00}{#1}}
\newcommand{\VerbatimStringTok}[1]{\textcolor[rgb]{0.31,0.60,0.02}{#1}}
\newcommand{\WarningTok}[1]{\textcolor[rgb]{0.56,0.35,0.01}{\textbf{\textit{#1}}}}
\usepackage{graphicx}
\makeatletter
\def\maxwidth{\ifdim\Gin@nat@width>\linewidth\linewidth\else\Gin@nat@width\fi}
\def\maxheight{\ifdim\Gin@nat@height>\textheight\textheight\else\Gin@nat@height\fi}
\makeatother
% Scale images if necessary, so that they will not overflow the page
% margins by default, and it is still possible to overwrite the defaults
% using explicit options in \includegraphics[width, height, ...]{}
\setkeys{Gin}{width=\maxwidth,height=\maxheight,keepaspectratio}
% Set default figure placement to htbp
\makeatletter
\def\fps@figure{htbp}
\makeatother
\setlength{\emergencystretch}{3em} % prevent overfull lines
\providecommand{\tightlist}{%
  \setlength{\itemsep}{0pt}\setlength{\parskip}{0pt}}
\setcounter{secnumdepth}{-\maxdimen} % remove section numbering
\ifLuaTeX
  \usepackage{selnolig}  % disable illegal ligatures
\fi
\IfFileExists{bookmark.sty}{\usepackage{bookmark}}{\usepackage{hyperref}}
\IfFileExists{xurl.sty}{\usepackage{xurl}}{} % add URL line breaks if available
\urlstyle{same} % disable monospaced font for URLs
\hypersetup{
  pdftitle={Projekt R markdown},
  pdfauthor={Kacper Ludwiczak},
  hidelinks,
  pdfcreator={LaTeX via pandoc}}

\title{Projekt R markdown}
\author{Kacper Ludwiczak}
\date{2023-03-10}

\begin{document}
\maketitle

Regression Model Projekt - Kacper Ludwiczak, 221303

Temat: Kształtowanie się oczekiwanej długości życia dla większości
państw świata na podstawie modelu regresji i czynniki wpływające na jego
wysokość. Zaczynając projekt pracowałem na danych związanych z
nowoczesnymi technologiami. Dokonałem czyszczenia danych, obliczeń w
excelu oraz obliczeń w języku R. Projekty znajdują się pod nazwą ``Stary
projekt - Reggresion Models'', ``Stary projekt -- Zestawienie'', ``Stary
projekt w R''. Jednakże stwierdziłem, że wyniki nie są zadowalające i
porzuciłem ten projekt. Następnie zacząłem prace nad tym modelem. Dane
pozyskałem ze strony kaggle.com. Według strony dane pochodzą ze strony
internetowej WHO i ONZ z pomocą Deeksha Russella i Duana Wanga. Dane
dotyczą większości państw świata, ich różnych zbiorów informacji, np.
Długość życia. Plik z oryginałem danych jest pod nazwą ``Life Expectancy
Data''. Po pierwsze chciałem skupić się na próbie 30 krajów. Wybrałem
zatem 30 krajów, metodą losową. Dane obliczyłem zarówno w excelu jak i w
języku R. Dane i obliczenia znajdują się w pliku ``Projekt w R próba''
oraz ``Projekt w Excel próba''. Uznałem, że lepszym pomysłem będzie
stworzenie modelu z większości krajów na jakich dano było mi pracować.
Zacząłem od pozostawienia tylko krajów z 2015 jako najbardziej
aktualnych z danej bazy. Dane są pod nazwą ``Dane z 2015''. Następnie
zamieniłem dane za pomocą funkcji ``Tekst jako kolumny''. Usunąłem
kolumny ``Year'', ``Status'', ponieważ kolumny są mi niepotrzebne.
Usunąłem kolumny ``Alcohol'', ``percentage expenditure'', ``Total
expenditure'', ponieważ były zauważające braki danych. Usunąłem kolumny
``BMI'', ``HIV/AIDS'', ``thinness 1-19 years'', ``thinness 5-9 years'',
``Schooling'', ponieważ pojawiła mi się pewna anomalia. Podczas zamiany
na kolumny niektóre dane są formacie daty, po zamienieniu ich na liczby
pojawiały się błędne liczby. Również nie które liczby zamienione
automatycznie są błędne. Po wielu nieudanych próbach rozwiązania tego
problemu, dokonałem eliminacji tych kolumn. Również musiałem dokonać
zamiany kropek na przecinki w liczbach oraz wypełnić puste pola średnią
z reszty kolumny. Dane znajdują się w pliku ``Dane zrobione''. Również
stworzyłem osobny plik dla importowania do RStudio, pod nazwą ``Dane do
R''. Szczegóły dotyczące danych: • Country - Kraj • Year - Rok • Status
- Stan rozwinięty lub rozwijający się • Life expectancy - Oczekiwana
długość życia w latach • Adult Mortality - Wskaźniki śmiertelności
dorosłych obu płci (prawdopodobieństwo śmierci w wieku od 15 do 60 lat
na 1000 mieszkańców) • Infant deaths - Liczba zgonów niemowląt na 1000
ludności • Alcohol - Spożycie alkoholu na mieszkańca (15+) (w litrach
czystego alkoholu) • Hepatitis B - Zasięg szczepień przeciw wirusowemu
zapaleniu wątroby typu B (HepB) wśród 1-latków (\%) • Measles -- Odra,
liczba zgłoszonych przypadków na 1000 ludności • BMI - Średni wskaźnik
masy ciała całej populacji • Under-five deaths - Liczba zgonów poniżej
piątego roku życia na 1000 mieszkańców • Polio - Zasięg szczepień
przeciw polio (Pol3) wśród 1-latków (\%) • Total expenditure - Wydatki
sektora instytucji rządowych i samorządowych na zdrowie jako odsetek
wydatków sektora instytucji rządowych i samorządowych ogółem (\%) •
Diphtheria - Odsetek szczepień przeciwko anatoksynie błonicy i tężcowi
oraz krztuścowi (DTP3) wśród 1-latków (\%) • HIV/AIDS - Zgony na 1000
żywych urodzeń HIV/AIDS (0-4 lata) • GDP - Produkt Krajowy Brutto per
capita (w USD) • Population - Ludność kraju • Thinness 1-19 years -
Rozpowszechnienie szczupłości wśród dzieci i młodzieży w wieku od 10 do
19 lat (\%) • Thinness 5-9 years - Występowanie szczupłości wśród dzieci
w wieku od 5 do 9 lat (\%) • Income - Wskaźnik rozwoju społecznego pod
względem struktury dochodów zasobów (wskaźnik w zakresie od 0 do 1) •
Schooling- Liczba lat nauki (lata)

Projekt w RStudio jest pod nazwą ``Projekt R markdown''

Zainstalowałem pakiet ``readxl'', oraz uruchomiłem go przez funkcje
``library''.

\begin{Shaded}
\begin{Highlighting}[]
\FunctionTok{library}\NormalTok{(readxl)}
\end{Highlighting}
\end{Shaded}

\begin{verbatim}
## Warning: pakiet 'readxl' został zbudowany w wersji R 4.2.2
\end{verbatim}

Kod wczytuje plik excel ``Dane do R.xlsx'' i zapisuje jego zawartość w
zmiennej ``Dane''. Następnie wyświetla zawartość tej zmiennej za pomocą
funkcji ``View()'', wyświetla nazwy kolumn danych za pomocą funkcji
``names()'', a na końcu wyświetla wartości kolumny ``Life expectancy''
za pomocą funkcji ``print()''.

\begin{Shaded}
\begin{Highlighting}[]
\NormalTok{Dane }\OtherTok{\textless{}{-}}  \FunctionTok{read\_excel}\NormalTok{(}\StringTok{"C:/Users/Kacper/Desktop/Projekt Regression/Dane do R.xlsx"}\NormalTok{)}
\FunctionTok{View}\NormalTok{(Dane)}
\FunctionTok{names}\NormalTok{(Dane)}
\end{Highlighting}
\end{Shaded}

\begin{verbatim}
##  [1] "Life expectancy" "Adult Mortality" "Infant deaths"   "Hepatitis B"    
##  [5] "Measles"         "Under-5 deaths"  "Polio"           "Diphtheria"     
##  [9] "GDP"             "Population"      "Income"
\end{verbatim}

\begin{Shaded}
\begin{Highlighting}[]
\FunctionTok{print}\NormalTok{(Dane}\SpecialCharTok{$}\StringTok{"Life expectancy"}\NormalTok{ )}
\end{Highlighting}
\end{Shaded}

\begin{verbatim}
##   [1] 65.0 77.8 75.6 52.4 76.4 76.3 74.8 82.8 81.5 72.7 76.1 76.9 71.8 75.5 72.3
##  [16] 81.1 71.0 60.0 69.8 77.0 77.4 65.7 75.0 77.7 74.5 59.9 59.6 73.3 68.7 57.3
##  [31] 82.2 52.5 53.1 85.0 76.1 74.8 63.5 64.7 79.6 78.0 79.1 85.0 78.8 76.0 59.8
##  [46] 86.0 63.5 73.9 76.2 79.0 73.5 58.2 64.7 77.6 64.8 69.9 81.1 82.4 66.0 61.1
##  [61] 74.4 81.0 62.4 81.0 73.6 71.9 59.0 58.9 66.2 63.5 74.6 75.8 82.7 68.3 69.1
##  [76] 75.5 68.9 81.4 82.5 82.7 76.2 83.7 74.1 72.0 63.4 66.3 74.7 71.1 65.7 74.6
##  [91] 74.9 53.7 61.4 72.7 73.6 82.0 65.5 58.3 75.0 78.5 58.2 81.7 63.1 74.6 76.7
## [106] 69.4 68.8 76.1 74.3 57.6 66.6 65.8 69.2 81.9 81.6 74.8 61.8 54.5 81.8 76.6
## [121] 66.4 77.8 62.9 74.0 75.5 68.5 77.5 81.1 78.2 82.3 72.1 75.0 75.0 66.1 75.2
## [136] 73.2 74.0 67.5 74.5 66.7 75.6 73.2 51.0 83.1 76.7 88.0 69.2 55.0 62.9 57.3
## [151] 82.8 74.9 64.1 71.6 58.9 82.4 83.4 64.5 69.7 74.9 75.7 68.3 59.9 73.5 71.2
## [166] 75.3 75.8 66.3 62.3 71.3 77.1 81.2 61.8 79.3 77.0 69.4 72.0 74.1 76.0 65.7
## [181] 61.8 67.0
\end{verbatim}

Tutaj za pomocą funkcji ``colnames'' dokonałem zmiany nazw kolumn w celu
szybszej pracy w dalszych etapach.

\begin{Shaded}
\begin{Highlighting}[]
\FunctionTok{colnames}\NormalTok{(Dane)}\OtherTok{\textless{}{-}}\FunctionTok{c}\NormalTok{(}\StringTok{"Life"}\NormalTok{,}\StringTok{"Adult\_M"}\NormalTok{,}\StringTok{"Infant\_D"}\NormalTok{,}\StringTok{"H\_B"}\NormalTok{,}\StringTok{"Measles"}\NormalTok{,}\StringTok{"Under\_D"}\NormalTok{,}\StringTok{"Polio"}\NormalTok{,}\StringTok{"Dipht"}\NormalTok{,}\StringTok{"GDP"}\NormalTok{,}\StringTok{"Popl"}\NormalTok{,}\StringTok{"Income"}\NormalTok{)}
\end{Highlighting}
\end{Shaded}

Użyłem funkcji ``plot'' w celu zobaczeniu na wykresie dane z kolumny
``Life''. Kolumna ``Life'' zostaje moją zmienną zależną, natomiast
reszta kolumn zmiennymi niezależnymi.

\begin{Shaded}
\begin{Highlighting}[]
\FunctionTok{plot}\NormalTok{(Dane}\SpecialCharTok{$}\StringTok{"Life"}\NormalTok{)}
\end{Highlighting}
\end{Shaded}

\includegraphics{Projekt-R-markdown_files/figure-latex/unnamed-chunk-4-1.pdf}

Sprawdziłem następujące wartości danych: • Minimalna wartość = 51.00 •
Pierwszy kwartal = 65.85 • Mediana = 73.95 • Średnia = 71.72 • Trzeci
kwartal = 76.97 • Maksymalna wartość = 88.00 , dla zmiennej zależnej.

\begin{Shaded}
\begin{Highlighting}[]
\FunctionTok{summary}\NormalTok{(Dane}\SpecialCharTok{$}\StringTok{"Life"}\NormalTok{ )}
\end{Highlighting}
\end{Shaded}

\begin{verbatim}
##    Min. 1st Qu.  Median    Mean 3rd Qu.    Max. 
##   51.00   65.85   73.95   71.72   76.97   88.00
\end{verbatim}

W celu zobrazowania graficznego użyłem funkcji ``hist''. Według tego
można zauważyć dominacje wartości w przedziale od 70 do 80.

\begin{Shaded}
\begin{Highlighting}[]
\FunctionTok{hist}\NormalTok{(Dane}\SpecialCharTok{$}\StringTok{"Life"}\NormalTok{ )}
\end{Highlighting}
\end{Shaded}

\includegraphics{Projekt-R-markdown_files/figure-latex/unnamed-chunk-6-1.pdf}
Za pomocą funkcji ``cor'' przedstawiłem korelacje wszystkich zmiennych.
Im dana liczba jest większa tym większa jest korelacja między
odpowiadającymi danymi w wierszu i kolumnie. Można z niej ustalić, że
korelacja między zmienną zależną ``Life'' a zmienną: • ``Adult\_M'' jest
duża ujemnie • ``Infant\_D'' jest bardzo mała ujemnie • ``H\_B'' jest
mała dodatnio • ``Measles'' jest bardzo mała ujemnie • ``Under\_D'' jest
bardzo mała ujemnie • ``Polio'' jest średnia dodatnio • ``Dipht'' jest
średnia dodatnio • ``GDP'' jest mała dodatnio • ``Popl'' jest bardzo
mała ujemnie • ``Income'' jest bardzo mała ujemnie

\begin{Shaded}
\begin{Highlighting}[]
\FunctionTok{cor}\NormalTok{(Dane)}
\end{Highlighting}
\end{Shaded}

\begin{verbatim}
##                 Life     Adult_M    Infant_D         H_B      Measles
## Life      1.00000000 -0.77215206 -0.23984305  0.40450050 -0.077810464
## Adult_M  -0.77215206  1.00000000  0.18604690 -0.23139939  0.054570792
## Infant_D -0.23984305  0.18604690  1.00000000 -0.08681527  0.801663443
## H_B       0.40450050 -0.23139939 -0.08681527  1.00000000  0.015656414
## Measles  -0.07781046  0.05457079  0.80166344  0.01565641  1.000000000
## Under_D  -0.27495588  0.21446481  0.99348313 -0.11001898  0.764990118
## Polio     0.52186030 -0.37773277 -0.12958871  0.59020367 -0.028359472
## Dipht     0.50655358 -0.32738699 -0.11788072  0.90925537 -0.001622602
## GDP       0.43089742 -0.31610931 -0.11953200  0.13188477 -0.076143290
## Popl     -0.04721323  0.03756295  0.26624483 -0.05672244  0.127925269
## Income   -0.13196089  0.13031897  0.02150182 -0.08670694  0.008493532
##              Under_D       Polio        Dipht         GDP         Popl
## Life     -0.27495588  0.52186030  0.506553583  0.43089742 -0.047213234
## Adult_M   0.21446481 -0.37773277 -0.327386992 -0.31610931  0.037562946
## Infant_D  0.99348313 -0.12958871 -0.117880715 -0.11953200  0.266244826
## H_B      -0.11001898  0.59020367  0.909255374  0.13188477 -0.056722444
## Measles   0.76499012 -0.02835947 -0.001622602 -0.07614329  0.127925269
## Under_D   1.00000000 -0.15154462 -0.143729736 -0.12620291  0.303720175
## Polio    -0.15154462  1.00000000  0.661623315  0.22349572 -0.221071926
## Dipht    -0.14372974  0.66162332  1.000000000  0.21333277 -0.065038775
## GDP      -0.12620291  0.22349572  0.213332775  1.00000000  0.046037030
## Popl      0.30372017 -0.22107193 -0.065038775  0.04603703  1.000000000
## Income    0.05419071 -0.06262976 -0.100745256 -0.04756283 -0.007033844
##                Income
## Life     -0.131960889
## Adult_M   0.130318973
## Infant_D  0.021501816
## H_B      -0.086706940
## Measles   0.008493532
## Under_D   0.054190707
## Polio    -0.062629761
## Dipht    -0.100745256
## GDP      -0.047562832
## Popl     -0.007033844
## Income    1.000000000
\end{verbatim}

Kod ten oblicza korelację między kolumnami w obiekcie ``Dane'', z
wyjątkiem pierwszej kolumny, która jest pomijana za pomocą wyrażenia
``{[}, -1{]}''. Jest to kolumna ``Life''. Funkcja ``cor'' oblicza
korelację Pearsona, czyli stopień zależności liniowej między dwoma
zmiennymi. Wynik jest następnie zaokrąglany do trzech miejsc po
przecinku za pomocą funkcji ``round''.

\begin{Shaded}
\begin{Highlighting}[]
\FunctionTok{round}\NormalTok{(}\FunctionTok{cor}\NormalTok{(Dane[,}\SpecialCharTok{{-}}\DecValTok{1}\NormalTok{]),}\DecValTok{3}\NormalTok{)}
\end{Highlighting}
\end{Shaded}

\begin{verbatim}
##          Adult_M Infant_D    H_B Measles Under_D  Polio  Dipht    GDP   Popl
## Adult_M    1.000    0.186 -0.231   0.055   0.214 -0.378 -0.327 -0.316  0.038
## Infant_D   0.186    1.000 -0.087   0.802   0.993 -0.130 -0.118 -0.120  0.266
## H_B       -0.231   -0.087  1.000   0.016  -0.110  0.590  0.909  0.132 -0.057
## Measles    0.055    0.802  0.016   1.000   0.765 -0.028 -0.002 -0.076  0.128
## Under_D    0.214    0.993 -0.110   0.765   1.000 -0.152 -0.144 -0.126  0.304
## Polio     -0.378   -0.130  0.590  -0.028  -0.152  1.000  0.662  0.223 -0.221
## Dipht     -0.327   -0.118  0.909  -0.002  -0.144  0.662  1.000  0.213 -0.065
## GDP       -0.316   -0.120  0.132  -0.076  -0.126  0.223  0.213  1.000  0.046
## Popl       0.038    0.266 -0.057   0.128   0.304 -0.221 -0.065  0.046  1.000
## Income     0.130    0.022 -0.087   0.008   0.054 -0.063 -0.101 -0.048 -0.007
##          Income
## Adult_M   0.130
## Infant_D  0.022
## H_B      -0.087
## Measles   0.008
## Under_D   0.054
## Polio    -0.063
## Dipht    -0.101
## GDP      -0.048
## Popl     -0.007
## Income    1.000
\end{verbatim}

Kod ten tworzy macierz z danych zawartych w obiekcie ``Dane''. Wiersze z
obiektu ``Dane'' są wczytywane za pomocą wyrażenia ``{[}, -1{]}'', co
oznacza, że pierwsza kolumna jest pomijana. Jest to kolumna ``Life''.
Następnie dane są konwertowane na macierz za pomocą funkcji
``as.matrix''. Ostatecznie macierz jest przypisywana do obiektu
``matrix''.

\begin{Shaded}
\begin{Highlighting}[]
\NormalTok{matrix }\OtherTok{\textless{}{-}} \FunctionTok{as.matrix}\NormalTok{(Dane[,}\SpecialCharTok{{-}}\DecValTok{1}\NormalTok{])}
\end{Highlighting}
\end{Shaded}

Kod oblicza własne wartości i wektory dla macierzy ``matrix''. Macierz
jest najpierw transponowana za pomocą funkcji ``t'', a następnie mnożona
przez siebie za pomocą operatora ``\%*\%``, co daje macierz kowariancji.
Otrzymaną macierz kowariancji jest analizowana przez funkcję''eigen'',
która oblicza własne wartości i wektory. Wynik jest przypisywany do
obiektu ``matrix\_eigen''. Dzięki temu mogę zobaczyć własne wartości
macierzy

\begin{Shaded}
\begin{Highlighting}[]
\NormalTok{matrix\_eigen }\OtherTok{\textless{}{-}} \FunctionTok{eigen}\NormalTok{(}\FunctionTok{t}\NormalTok{(matrix) }\SpecialCharTok{\%*\%}\NormalTok{ matrix)}
\NormalTok{matrix\_eigen}\SpecialCharTok{$}\NormalTok{val}
\end{Highlighting}
\end{Shaded}

\begin{verbatim}
##  [1] 1.365373e+17 2.887223e+10 1.122436e+10 6.501808e+06 1.227256e+06
##  [6] 7.446202e+05 4.555312e+04 1.774313e+04 7.863299e+03 5.388979e+03
\end{verbatim}

Kod wylicza pierwiastek kwadratowy elementów ilorazu dwóch wektorów za
pomocą ``sqrt''
``matrix\_eigen\(val[1]” jest pierwszym elementem wektora “matrix_eigen\)val'',
a ``matrix\_eigen\$val'' jest całym wektorem. Dostając następujące
wyniki. Wyniki powyżej 30 wskazują na brak współliniowości. W moim
przypadku można zauważyć, że współliniowość nie występuje. Na tym etapie
można wykluczyć jako by mój model był dobrym modelem.

\begin{Shaded}
\begin{Highlighting}[]
\FunctionTok{sqrt}\NormalTok{(matrix\_eigen}\SpecialCharTok{$}\NormalTok{val[}\DecValTok{1}\NormalTok{]}\SpecialCharTok{/}\NormalTok{matrix\_eigen}\SpecialCharTok{$}\NormalTok{val)}
\end{Highlighting}
\end{Shaded}

\begin{verbatim}
##  [1]       1.000    2174.631    3487.746  144913.411  333547.800  428211.415
##  [7] 1731277.043 2774026.198 4166997.596 5033527.140
\end{verbatim}

Sprawdziłem na wykresie jaka jest korelacje zmiennej zależnej i zmiennej
``Adult\_M'' w postaci graficznej. Tutaj przetestowałem funkcje ``lm''
za pomocą stworzenia modelu prostej regresji liniowej z użyciem jednej
zmiennej niezależnej. Funkcja ``abline'' umożliwiła mi dodanie
niebieskiej linii regresji, która najdokładniej próbuje dopasować się do
istniejących danych. Można zauważyć, że linia jest przekrzywiona w dół
po stronie lewej, wynika to z pojawiających się wartości odstających,
poniżej wartości 100 dla zmiennej niezależnej.

\begin{Shaded}
\begin{Highlighting}[]
\FunctionTok{plot}\NormalTok{(Dane}\SpecialCharTok{$}\StringTok{"Life"}\SpecialCharTok{\textasciitilde{}}\NormalTok{Dane}\SpecialCharTok{$}\StringTok{"Adult\_M"}\NormalTok{)}
\NormalTok{SimpleModel}\OtherTok{\textless{}{-}} \FunctionTok{lm}\NormalTok{(Dane}\SpecialCharTok{$}\StringTok{"Life"}\SpecialCharTok{\textasciitilde{}}\NormalTok{Dane}\SpecialCharTok{$}\StringTok{"Adult\_M"}\NormalTok{)}
\FunctionTok{abline}\NormalTok{(SimpleModel}\SpecialCharTok{$}\NormalTok{coef,}\AttributeTok{col=}\StringTok{"blue"}\NormalTok{)}
\end{Highlighting}
\end{Shaded}

\includegraphics{Projekt-R-markdown_files/figure-latex/unnamed-chunk-12-1.pdf}

Wynik p-value wyszedł bardzo dobry. Jednakże wartość R-squared zbyt
mała.

\begin{Shaded}
\begin{Highlighting}[]
\FunctionTok{summary}\NormalTok{(SimpleModel)}
\end{Highlighting}
\end{Shaded}

\begin{verbatim}
## 
## Call:
## lm(formula = Dane$Life ~ Dane$Adult_M)
## 
## Residuals:
##      Min       1Q   Median       3Q      Max 
## -21.2282  -1.6823   0.7664   2.9876  11.2813 
## 
## Coefficients:
##               Estimate Std. Error t value Pr(>|t|)    
## (Intercept)  81.492535   0.709494   114.9   <2e-16 ***
## Dane$Adult_M -0.064512   0.003957   -16.3   <2e-16 ***
## ---
## Signif. codes:  0 '***' 0.001 '**' 0.01 '*' 0.05 '.' 0.1 ' ' 1
## 
## Residual standard error: 5.117 on 180 degrees of freedom
## Multiple R-squared:  0.5962, Adjusted R-squared:  0.594 
## F-statistic: 265.8 on 1 and 180 DF,  p-value: < 2.2e-16
\end{verbatim}

Stworzenie pierwszego modelu. Używam metody ``Backward'' ręcznie.
Eliminując zmienne o najwyższym wskaźniku p-value, tak długo aż model
będzie zadawalający. Zakładam dla modelu, że poziom p-value poniżej 0.05
jest odpowiedni.

\begin{Shaded}
\begin{Highlighting}[]
\NormalTok{Model }\OtherTok{\textless{}{-}} \FunctionTok{lm}\NormalTok{(Life}\SpecialCharTok{\textasciitilde{}}\NormalTok{Adult\_M}\SpecialCharTok{+}\NormalTok{Infant\_D}\SpecialCharTok{+}\NormalTok{H\_B}\SpecialCharTok{+}\NormalTok{Measles}\SpecialCharTok{+}\NormalTok{Under\_D}\SpecialCharTok{+}\NormalTok{Polio}\SpecialCharTok{+}\NormalTok{Dipht}\SpecialCharTok{+}\NormalTok{GDP}\SpecialCharTok{+}\NormalTok{Popl}\SpecialCharTok{+}\NormalTok{Income,}\AttributeTok{data=}\NormalTok{Dane)}
\FunctionTok{summary}\NormalTok{(Model)}
\end{Highlighting}
\end{Shaded}

\begin{verbatim}
## 
## Call:
## lm(formula = Life ~ Adult_M + Infant_D + H_B + Measles + Under_D + 
##     Polio + Dipht + GDP + Popl + Income, data = Dane)
## 
## Residuals:
##      Min       1Q   Median       3Q      Max 
## -16.3833  -2.4605   0.2957   2.3275  11.8063 
## 
## Coefficients:
##               Estimate Std. Error t value Pr(>|t|)    
## (Intercept)  6.931e+01  1.761e+00  39.356  < 2e-16 ***
## Adult_M     -4.718e-02  3.959e-03 -11.919  < 2e-16 ***
## Infant_D     9.576e-02  4.644e-02   2.062 0.040728 *  
## H_B         -9.270e-03  3.379e-02  -0.274 0.784131    
## Measles      1.253e-05  7.851e-05   0.160 0.873375    
## Under_D     -8.295e-02  3.440e-02  -2.412 0.016940 *  
## Polio        5.024e-02  1.887e-02   2.662 0.008500 ** 
## Dipht        6.339e-02  3.886e-02   1.631 0.104718    
## GDP          1.202e-04  3.239e-05   3.712 0.000278 ***
## Popl         2.531e-08  1.461e-08   1.732 0.085120 .  
## Income       1.122e-02  3.531e-02   0.318 0.751148    
## ---
## Signif. codes:  0 '***' 0.001 '**' 0.01 '*' 0.05 '.' 0.1 ' ' 1
## 
## Residual standard error: 4.346 on 171 degrees of freedom
## Multiple R-squared:  0.7233, Adjusted R-squared:  0.7071 
## F-statistic: 44.69 on 10 and 171 DF,  p-value: < 2.2e-16
\end{verbatim}

\end{document}
